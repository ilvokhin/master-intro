\documentclass[12pt]{report}

\usepackage{fullpage}
\usepackage{multicol, multirow}
\usepackage{tabularx}
\usepackage{standalone}
\usepackage{listings}
\usepackage{ulem}
\usepackage{amsmath}
\usepackage{amssymb}
\usepackage{pdfpages}
\usepackage[T1, T2A]{fontenc}
\usepackage[utf8]{inputenc}
\usepackage[russian]{babel}

\usepackage{indentfirst}
% setup  page fields
\usepackage{geometry}
\geometry{left=25mm}
\geometry{right=20mm}
\geometry{top=20mm}
\geometry{bottom=20mm}

\usepackage{titlesec}
\titleformat{\chapter}{\filcenter\bfseries}{\thechapter.}{1em}{}
\titleformat{\section}{\filcenter\bfseries}{\thesection}{1em}{}
\titleformat{\subsection}{\filcenter\bfseries}{\thesubsection}{1em}{}
\titleformat{\subsubsection}{\filcenter\bfseries}{\thesubsubsection}{1em}{}
\titleformat{\paragraph}{\filcenter\bfseries}{\paragraph}{1em}{}

\usepackage{tocloft}
% add dots for chapters
\renewcommand{\cftchapleader}{\cftdotfill{\cftdotsep}}
\renewcommand\cftchapfont{\mdseries}
\renewcommand\cftchappagefont{\mdseries}
% add dot after chapter number
\renewcommand{\cftchapaftersnum}{.}
% clickable toc
\usepackage{hyperref}

%
\usepackage{amsthm}
\theoremstyle{plain}
\newtheorem{theorem}{Теорема}[chapter]
\newtheorem{definition}[theorem]{Определение}
\newtheorem{statement}[theorem]{Утверждение}

% Some usefull macro:
\newcommand{\R}{\mathbb R}
\newcommand{\N}{\mathbb N}
\newcommand{\Ri}{\mathcal R}
\newcommand{\B}{\mathcal B}

\begin{document}

\tableofcontents

\chapter{Математический анализ}

\section{Аксиоматика и общие свойства множества действительных чисел. Полнота множества действительных чисел.}
\begin{definition}
Множество $\R$ называется множеством действительных (вещественных) чисел,
а его элементы действительными (вещественными) числами, если выполняются следующие
условия (аксиомы действительных чисел):
\end{definition}

\subsection{Аксиомы сложения}
Определено отображение (операция сложения):
$$
  +: \R \times \R \rightarrow \R,
$$
которое каждой упорядоченной паре $(x, y)$ ставит в соответствие элемент
$x + y \in \R$, называемый суммой $x$ и $y$.
При этом выполнены следующие условия:

\begin{enumerate}

\item Существует нейтральный элемент $0$ (в случае сложения называется нулем) такой,
что для любого $x\in \R$:
$$
x + 0 = 0 + x = x.
$$

\item Для любого элемента $x \in \R$ имеется элемент $-x \in \R$,
называемый противоположным элементом к $x$, такой, что:
$$
x + (-x) = (-x) + x = 0.
$$

\item Операция $+$ --- ассоциативна, т.е. для любых $x, y, z \in \R$ выполнено:
$$
(x + y) + z = x + (y + z).
$$

\item Операция $+$ --- коммутативна, т.е. для любых $x, y \in \R$ выполнено:
$$
x + y = y + x.
$$

\end{enumerate}

Если на каком-то множестве $G$ выполнены аксиомы 1-3, говорят, что $G$ --- группа.
Если операцию называют сложением, то группа называется аддитивной.
Если известно, что операция коммутативна (т.е. выполнена аксиома 4), говорят, что
группа коммутативна или абелева.

\subsection{Аксиомы умножения}
Определено отображение (операция умножения):
$$
\cdot: \R \times \R \rightarrow \R,
$$
составляющая каждой упорядоченной паре $(x, y) x, y \in \R$
некоторый элемент $x \cdot y \in \R$, называемый произведением $x$ и $y$,
причем так, что выполнены следующие условия:

\begin{enumerate}
\item Существует нейтральный элемент $1 \in R \setminus 0$,
в случае умножения называемый единицей, так что
$$
x \cdot 1 = 1 \cdot x = x.
$$

\item Для $\forall x \in \R \setminus 0$ имеется элемент $x^{-1}$, называемый
обратным, такой, что:
$$
x \cdot x^{-1} = x^{-1} \cdot x = 1.
$$

\item Операция $\cdot$ --- ассоциативна, т.е. для любых $x, y, z \in \R$:
$$
x\cdot (y \cdot z) = (x \cdot y) \cdot z.
$$

\item Операция $\cdot$ --- коммутативна, т.е. для $x, y \in \R$:
$$
x \cdot y = y \cdot x.
$$

\end{enumerate}

Связь сложения и умножения: умножение дистрибутивно по отношению к сложению, т.е.
$\forall x, y, z \in \R$:
$$
(x + y) \cdot z = xz + xy.
$$

Если на множестве $G$ действуют две операции, удовлетворяющие всем перечисленным выше аксиомам,
то $G$ называется алгебраическим полем или просто полем.

\subsection{Аксиомы порядка}
Между элементами $\R$ имеется отношение $\le$, т.е. для элементов $x, y \in \R$ установлено,
выполняется ли $x \le y$ или нет. При этом должны удовлетворятся следующие условия:

\begin{enumerate}
\item $\forall x \in \R: x \le x$.
\item $(x \le y) \land (y \le x) \implies (x = y)$.
\item $(x \le y) \land (y \le z) \implies (x \le z)$.
\item $\forall x \in \R, \forall y \in \R: (x \le y) \lor (y \le x)$.
\end{enumerate}

Отношение $\le$ называется отношением неравенства в $\R$.

Множество, в котором выполняются аксиомы 1-3 называют частично упорядоченным.
Если сверх того выполнена аксиома 4, т.е. любые два элемента множества сравнимы, то
множество называется линейно упорядоченным.

Связь сложения и порядка в $\R$: $\forall x, y, z \in \R: (x \le y) \implies (x + z \le y + z)$.
Связь умножения и порядка в $\R$: $(0 \le x) \land (0 \le y) \implies (0 \le x \cdot y)$.

\subsection{Аксиома (полноты) непрерывности}
Если $X$ и $Y$ --- непустые подмножества $\R$, обладающие тем свойством,
что для любых элементов $x \in X$ и $y \in Y$ выполнено $x \le y$,
то существует такое $c: x \le c \le y$ для любых элементов $x \in X$ и $y \in Y$.

\subsection{Общие свойства множества действительных чисел}

\subsubsection{Следствия из аксиом сложения}
\begin{enumerate}
\item В множестве действительных чисел имеется только один нуль.
\item В множестве действительных чисел у каждого элемента имеется
  только один противоположный элемент.
\item Уравнение $a + x = b$ в $\R$ имеет и притом единственное решение:
  $x = b + (-a)$.
\end{enumerate}

\subsubsection{Следствия из аксиом умножения}
\begin{enumerate}
\item В множестве действительных чисел имеется только одна единица.
\item Для каждого числа $x \ne 0$ имеется только один обратный элемент $x^{-1}$.
\item Уравнение $a \cdot x = b$ при $a \in R \setminus 0$ имеет и притом единственное
  решение $x = b \cdot a^{-1}$.
\end{enumerate}

\subsubsection{Следствия из аксиом связи сложения и умножения}
\begin{enumerate}
\item $\forall x \in \R: x \cdot 0 = 0 \cdot x = 0$.
\item $(x \cdot y = 0) \implies (x = 0) \lor (y = 0)$.
\item $\forall x \in \R: -x = (-1) \cdot x$.
\item $\forall x \in \R: (-1) \cdot (-x) = x$.
\item $\forall x \in \R: (-x) \cdot (-x) = x \cdot x$.
\end{enumerate}

\subsubsection{Следствия аксиом порядка}
\begin{enumerate}
\item $\forall x, y \in \R$ всегда имеет место в точности одно из отношений:
$$
x < y, \quad x = y, \quad x > y.
$$
\item $\forall x, y, z \in \R$:
  \begin{gather*}
    (x < y) \land (y \le z) \implies (x < z), \\
    (x \le y) \land (y < z) \implies (x < z).
  \end{gather*}
\end{enumerate}

\subsubsection{Следствия аксиом связи порядка со сложением и умножением}
\begin{enumerate}
\item $\forall x, y, z, w \in \R$:
  \begin{gather*}
    (x < y) \implies (x + z) < (y + z), \\
    (0 < x) \implies (-x < 0), \\
    (x \le y) \land (z \le w) \implies (x + z \le y + w), \\
    (x \le y) \land  (z < w) \implies (x + z < y + w).
  \end{gather*}
\item $\forall x, y, z \in \R$:
  \begin{gather*}
    (0 < x) \land (0 < y) \implies (0 < xy), \\
    (x < 0) \land (y < 0) \implies (0 < xy), \\
    (x < 0) \land (0 < y) \implies (xy < 0), \\
    (x < y) \land (0 < z) \implies (xz < yz), \\
    (x < y) \land (z < 0) \implies (yz < xz).
  \end{gather*}
\item $0 < 1$.
\item $(0 < x) \implies (0 < x^{-1})$ и $(0 < x) \land (x < y) \implies (0 < y^{-1}) \land (y^{-1} < x^{-1}).$
\end{enumerate}

% Источник: Зорич, страницы 41 - 50.
% Там можно найти доказательства некоторых следствий.

\section{Предел последовательности. Предел функции. Их свойства.}
\subsection{Предел последовательности}
\begin{definition}
Число $A$ называется пределом числовой последовательности $\{x_n\}$,
если для любой окрестности $V(A)$ точки $A$, существует такой номер $N$
(выбираемый в зависимости от $V(A)$), что все члены последовательности,
с номерами больше $N$ лежат в указанной окрестности точки $A$.
\end{definition}

Другое, более распространенное определение:

\begin{definition}
Число $A \in \R$ называется пределом последовательности $\{x\}_n$, если
для $\forall \varepsilon > 0$ существует номер $N$, такой, что при всех
$n > N$ имеем $|x_n - A| < \varepsilon$.
\end{definition}

Последнее определение в логической символике выглядит следующим образом:
$$
  \lim\limits_{n \rightarrow \infty} x_n = A
    \Leftrightarrow
  \forall \varepsilon > 0, \exists N \in \N: \forall n > N: |x_n - A| < \varepsilon.
$$

Последовательность, имеющая предел называется сходящейся.
Последовательность, не имеющая предела, называется расходящейся.

\subsection{Свойства предела последовательности}

\subsubsection{Общие свойства}
Последовательность, принимающая только одно значение называется постоянной.

\begin{definition}
Если существует число $A$ и номер $N$, такие, что
$x_n = A$, при $\forall n > N$, то последовательность
$\{x_n\}$ называется финально постоянной.
\end{definition}

\begin{definition}
Последовательность $\{x_n\}$ называется ограниченной, если существует
такое число $M$, что $|x_n| < M, \forall n \in \N$.
\end{definition}

\begin{theorem}
  \begin{enumerate}
  \item Финально постоянная последовательность сходится.
    \item Любая окрестность предела последовательности содержит все члены последовательности,
    за исключением конечного их числа.
    \item Последовательность не может иметь двух различных пределов.
    \item Сходящаяся последовательность ограничена.
  \end{enumerate}
\end{theorem}

\subsubsection{Предельный переход и арифметические операции}

\begin{definition}
Если $\{x_n\}$ и $\{y_n\}$ --- числовые последовательности, то их суммой,
произведением, частным называются соответственно последовательности:
$$
\{(x_n + y_n)\}, \{(x_n \cdot y_n)\}, \left\{\dfrac{x_n}{y_n}\right\}.
$$
Частое определено при $\{y_n\} \ne 0$.
\end{definition}

\begin{theorem}
Пусть $\{x_n\}$ и $\{y_n\}$ --- числовые последовательности.
Если $A = \lim\limits_{n\rightarrow\infty} x_n, B = \lim\limits_{n\rightarrow\infty} y_n$, то
\begin{enumerate}
  \item $\lim\limits_{n\rightarrow\infty} x_n + y_n = A + B$.
  \item $\lim\limits_{n\rightarrow\infty} x_n \cdot y_n = A \cdot B$.
  \item $\lim\limits_{n\rightarrow\infty} x_n / y_n = A / B$, если $y_n \ne 0\, (n = 1, 2, \dots), B \ne 0$.
\end{enumerate}
\end{theorem}

\subsubsection{Предельный переход и неравенства}

\begin{theorem}
\begin{enumerate}
  \item Пусть $\{x_n\}$ и $\{y_n\}$ --- две сходящиеся последовательности, причем
    $\lim\limits_{n\rightarrow\infty} x_n = A, \lim\limits_{n\rightarrow\infty} y_n = B.$
    Если $A < B$, то найдется такой номер $N \in \N$, что при любом $n > N$ будет выполнятся
    $x_n < y_n$.
  \item Пусть последовательности $\{x_n\}$, $\{y_n\}$, $\{z_n\}$ таковы, что $x_n \le y_n \le z_n, \forall n \in \N$.
    Если при этом $\{x_n\}$, $\{z_n\}$ сходятся к одному и тому же приделу, то последовательность
    $\{y_n\}$ так же сходится к этому пределу.
\end{enumerate}
\end{theorem}

\subsection{Предел функции}
\begin{definition}
Точка $x$ называется предельной точкой множества $A$, если всякая проколотая окрестность точки
$x$ имеет с $A$ непустое пересечение.
\end{definition}

\begin{definition}
Пусть $E$ --- некоторое подмножество множества $\R$, а $a$ --- предельная точка множества
$E$. Пусть $f: E \rightarrow \R$ --- вещественная функция, определенная на $E$.


Тогда $A$ является пределом функции $f$ при $x$ стремящемся к $a$, если для
$\forall \varepsilon > 0, \exists \delta > 0$, такое что для любой точки 
$x \in E$ такой, что $ 0 < |x - a| < \delta$, выполнено отношение $|f(x) - A| < \varepsilon$.
\end{definition}

В логической символике определение выглядит следующим образом:
$$
\forall \varepsilon > 0, \exists \delta > 0: \forall x \in E \quad (0 < |x - a| < \delta \Rightarrow |f(x) - A| < \varepsilon).
$$

\subsection{Свойства предела функции}
\subsubsection{Общие свойства предела функции}
\begin{definition}
  Функцию $f: E \rightarrow R$ принимающая только одно значение называется постоянной.
\end{definition}

\begin{definition}
  Функцию $f: E \rightarrow R$ называется финально постоянной если она постоянна в некоторой
  проколотой окрестности точки $a$, предельной для множества $E$.
\end{definition}

\begin{theorem}
  \begin{enumerate}
    \item $f: E \rightarrow \R$ при $E \ni x \rightarrow a$, есть финально постоянная
      $A \Rightarrow \lim\limits_{x\rightarrow a} f(x) = A$.
    \item $\exists \lim\limits_{x\rightarrow a} f(x) = A \Rightarrow f$ --- финально ограничена при $E \ni x \rightarrow a$.
    \item $\lim\limits_{x\rightarrow a} f(x) = A_1 \land \lim\limits_{x\rightarrow a} f(x) = A_2 \Rightarrow A_1 = A_2$.
  \end{enumerate}
\end{theorem}

\subsection{Предельный переход и арифметические операции}
\begin{definition}
  Если две функции $f: E \rightarrow \R$, $g: E \rightarrow \R$, то их суммой, произведением и частным
  называются функции:
  \begin{gather*}
    (f + g)(x) = f(x) + g(x), \\
    (f \cdot g)(x) = f(x) \cdot g(x), \\
    \left(\dfrac{f}{g}(x)\right) = \dfrac{f(x)}{g(x)}, g(x) \ne 0, x \in E.
  \end{gather*}
\end{definition}

\begin{theorem}
  Пусть две функции $f: E \rightarrow \R$, $g: E \rightarrow \R$ с общей областью определения.
  Если $\lim\limits_{x\rightarrow a} f(x) = A$ и $\lim\limits_{x\rightarrow a} g(x) = B$, то
  \begin{enumerate}
    \item $\lim\limits_{x\rightarrow a} (f + g)(x) = A + B$.
    \item $\lim\limits_{x\rightarrow a} (f \cdot g)(x) = A \cdot B$.
    \item $\lim\limits_{x\rightarrow a} \dfrac{f}{g}(x) = \dfrac{A}{B}$, если $g(x) \ne 0, B \ne 0$.
  \end{enumerate}
\end{theorem}

\begin{definition}
  Функцию $f: E \rightarrow \R$ называют бесконечно малой при $E \ni x \rightarrow a$,
  если $\lim\limits_{x\rightarrow a} f(x) = 0$.
\end{definition}

\begin{theorem}
\begin{enumerate}
  \item Если $\alpha: E \rightarrow \R$ и $\beta: E \rightarrow \R$ --- бесконечно малые функции,
    при $x \rightarrow a$, то их сумма $\alpha + \beta$ --- тоже бесконечно малая функция при $x \rightarrow a$.
  \item Если $\alpha: E \rightarrow \R$ и $\beta: E \rightarrow \R$ --- бесконечно малые функции,
    при $x \rightarrow a$, то их произведение $\alpha \cdot \beta$ --- тоже бесконечно малая функция при $x \rightarrow a$.
  \item Если $\alpha: E \rightarrow \R$ --- бесконечно малая функция,
    а $\beta: E \rightarrow \R$ --- финально ограниченная функция,
    при $x \rightarrow a$, то их произведение $\alpha \cdot \beta$ --- бесконечно малая функция при $x \rightarrow a$.
\end{enumerate}
\end{theorem}

\subsubsection{Предельный переход и неравенства}
\begin{theorem}
  \begin{enumerate}
    \item  Если две функции $f: E \rightarrow \R$, $g: E \rightarrow \R$ таковы,
      что $\lim\limits_{x\rightarrow a} f(x) = A$ и $\lim\limits_{x\rightarrow a} g(x) = B$ и
      $A < B$, то найдется проколотая окрестность $U$ точки $a$ в множестве $E$, в любой
      точке которой выполнено неравенство $f(x) < g(x)$.
    \item  Если между функциями  $f: E \rightarrow \R$, $g: E \rightarrow \R$, $g: E \rightarrow \R$,
      имеет место соотношение $f(x) \le g(x) \le h(x)$ и
      $\lim\limits_{x\rightarrow a} f(x) = \lim\limits_{x\rightarrow a} h(x) = C$,
      то $\exists \lim\limits_{x\rightarrow a} g(x)$ и он равен $C$.
  \end{enumerate}
\end{theorem}

% Источники:
% 1. Зорич, страницы 92-133
% 2. https://ru.wikipedia.org/wiki/Предельная_точка

\section{Непрерывность функции. Свойства непрерывных функций. Теорема Вейерштрасса о максимальном значении.}

Описательно говоря, функция $f$ --- непрерывна в точке $a$, если ее значение $f(x)$ по мере приближения
аргумента $x$ к точке приближается к значению $f(a)$ функции в самой точке $a$.

\begin{definition}
Функция $f$ называется непрерывной в точке $a$, если для любой
окрестности $V(f(a))$ значения $f(a)$ функции точки $a$, найдется такая
окрестность $U(a)$ точки $a$, образ которой при отображении $f$ содержится
в $V(f(a))$.
\end{definition}

Формально-логическая запись определения:
$$
\forall \varepsilon > 0, \exists \delta > 0:
|x - a| < \delta \Rightarrow |f(x) - f(a)| < \varepsilon.
$$


\subsection{Свойства непрерывных функций}
\subsubsection{Локальный свойства}
Локальный свойства --- определяются поведением функции в сколь угодно
малой окрестности точки области определения.

\begin{theorem}
Пусть $f: E \rightarrow \R$ --- непрерывная в точке $a \in E$ функция.
Тогда справедливы следующие утверждения:
\begin{enumerate}
  \item Функция $f$ --- ограничена в некоторой окрестности $U_E(a)$ точки $a$.
  \item Если $f(a) \ne 0$ --- то в некоторой окрестности $U_E(a)$ точки $a$,
    все значения функции положительны или отрицательны вместе с $f(a)$.
  \item Если функция $g: U_E(a) \rightarrow \R$ определена в некоторой окрестности
    точки $a$, и, как и $f$ --- непрерывна в точке $a$, тогда
    \begin{enumerate}
      \item $(f + g)(x) := f(x) + g(x)$,
      \item $(f \cdot g)(x) := f(x) \cdot g(x)$,
      \item $\dfrac{f}{g}(x) := dfrac{f(x)}{g(x)}$, $(g(x) \ne 0)$
    \end{enumerate}
    определены в некоторой окрестности точки $a$ и непрерывны в ней.
  \item Если функция $g: Y \rightarrow \R$ непрерывна в точке $b \in Y$,
    а функция $f: E \rightarrow Y$, $f(a) = b$ и непрерывна в точке $a$,
    то композиция $(g \circ f)$ определена на $E$ и так же непрерывна в точке $a$.
\end{enumerate}
\end{theorem}

\subsubsection{Глобальные свойства}
Глобальные --- свойства, связанные со всей областью определения функции.

\begin{theorem}[Теорема Больцано-Коши о промежуточном значении]
Если функция непрерывная на отрезке принимает на его концах
значения разных знаков, то на отрезке есть точка, в которой
функция обращается в нуль.

В логической символике:
$$
f \in [a, b] \land f(a) \cdot f(b) < 0 \Rightarrow \exists c \in [a, b]: f(c) = 0.
$$
\end{theorem}

\subsection{Теорема Вейерштрасса о максимальном значении}
\begin{theorem}[Теорема Вейерштрасса о максимальном значении]
Функция, непрерывная на отрезке, ограничена на нем.
При этом на отрезке есть точка, где функция принимает максимальное
значение и есть точка, где функция принимает минимальное значение.
\end{theorem}

% Источники:
% 1. Зорич, страницы 175-188

\section{Производная и дифференциал. Основные правила дифференцирования. Формула Тейлора. Исследование функций.}

\begin{definition}
Функция $f: E \rightarrow \R$, называется дифференцируемой в точке
$x \in E$, предельной для множества $E$, если
$$
f(x + h) - f(x) = A(x) \cdot h + \alpha(x; h), \alpha(x; h) = o(h), h \rightarrow 0, x + h \in E.
$$ 
\end{definition}

$\Delta x(h) = (x + h)$ --- приращение аргумента.
$\Delta f(x; h) = f(x + h) - f(x)$ --- приращение функции.

\begin{definition}
Линейная по $h$ функция $h \mapsto A(x) \cdot h$ --- называется дифференциалом
функции $f: E \rightarrow \R$ в точке $x \in E$ и обозначается $df(x)$.
\end{definition}

% Действительно ли это так?
\begin{definition}
$A(x) = f'(x) = \lim\limits_{h\rightarrow 0,\, x+h, x\in E} \dfrac{f(x + h) - f(x)}{h}$ --- производная
функции $f$ в точке $x$.
\end{definition}

\subsection{Основные правила дифференцирования}
\begin{theorem}
Если функции $f: X \rightarrow \R$, $g: X \rightarrow \R$ --- дифференцируемы
в точке $x \in X$, то
\begin{enumerate}
  \item Их сумма дифференцируема в $x$, причем $(f + g)'(x) = (f' + g')(x)$.
  \item Их произведение дифференцируемо в $x$, причем $(f \cdot g)' = f'(x) g(x) + f(x) g'(x)$.
  \item Их отношение дифференцируемо в $x$, причем $\left(\dfrac{f}{g}(x)\right)' = \dfrac{f'(x) g(x) - f(x) g'(x)}{g^2(x)}$.
\end{enumerate}
\end{theorem}

\begin{theorem}
Если функция $f: X \rightarrow Y$ --- дифференцируема в точке $x \in X$,
a функция $g: Y \rightarrow \R$ --- дифференцируема в точке $y = f(x) \in Y$,
тогда композиция $g \circ f: X \rightarrow \R$ --- дифференцируема в точке $x$,
причем $d (g \circ f) = dg(y) \circ df(x)$ дифференциалов.
\end{theorem}

\begin{theorem}
Производная $(g \circ f)'(x)$ композиции дифференцируемых функций равна
произведению $g'(f(x)) \cdot f'(x)$ производных этих функций в соответствующих точках.
\end{theorem}

\subsection{Формула Тейлора}

\begin{definition}
Алгебраический полином, заданный соотношением:
$$
P_n(x, x_0) = P_n(x) = f(x_0) + \dfrac{f'(x_0)}{1!}(x - x_0) + \dfrac{f''(x_0)}{2!}(x - x_0)^2
+ \dots + \dfrac{f^{(n)}(x_0)}{n!} (x - x_0)^n
$$
называется полиномом Тейлора порядка $n$ функции $f(x)$ в точке $x_0$.
\end{definition}

\begin{definition}
Величина $f(x) - P_n(x_0, x) = r_n(x_0, x)$ --- уклонение
полинома $P_n$ от функции $f(x)$, называемое часто остатком
или $n$-м остатоком или $n$-м остаточным членом формулы Тейлора.
\end{definition}

Таким образом формула Тейлора:
$$
f(x) = f(x_0) + \dfrac{f'(x_0)}{1!}(x - x_0)+ \dots + 
\dfrac{f^{(n)}(x_0)}{n!} (x - x_0)^n + r(x, x_0)
$$

\subsubsection{Разложения некоторых функций}
\begin{enumerate}
\item $e^x$:
  $$
    e^x = 1 + \dfrac{1}{1!} x + \dfrac{1}{2!} x^2 + \dots + \dfrac{1}{n!} x^n + r_n(x, 0)
  $$
\item $a^x, 0 < a, a \ne 1$:
  $$
    a^x = 1 + \dfrac{\ln a}{1!} x + \dfrac{\ln^2 a}{2!} x^2 + \dots + \dfrac{\ln^n a}{n!} x^n + r_n(x, 0)
  $$
\item $sin(x)$:
  $$
    sin(x) = x - \dfrac{1}{3!} x^3 + \dfrac{1}{5!} x^5 - \dots + \dfrac{(-1)^n}{(2n + 1)!} x^{2n+1} + r_n(x, 0)
  $$
\item $cos(x)$:
  $$
    cos(x) = 1 - \dfrac{1}{2!} x^2 + \dfrac{1}{4!} x^4 - \dots + \dfrac{(-1)^n}{2n!} x^{2n} + r_n(x, 0)
  $$
\end{enumerate}

\subsection{Исследование функций}
\begin{statement}
Между характером монотонности дифференцируемой на интервале $E = (a, b)$
функции $f(x): E \rightarrow \R$ и знаком (положительностью) ее производной
$f'$ на этом интервале имеется следующая взаимосвязь:
\begin{enumerate}
\item $f'(x) > 0 \Rightarrow f(x) \text{ возрастает} \Rightarrow f'(x) \ge 0$
\item $f'(x) \ge 0 \Rightarrow f(x) \text{ не убывает} \Rightarrow f'(x) \ge 0$
\item $f'(x) \equiv 0 \Rightarrow f \equiv const \Rightarrow f'(x) = 0$
\item $f'(x) \le 0 \Rightarrow f(x) \text{ не возрастает} \Rightarrow f'(x) \le 0$
\item $f'(x) < 0 \Rightarrow f(x) \text{ убывает} \Rightarrow f'(x) \le 0$
\end{enumerate}
\end{statement}

\begin{statement}[Необходимые условия внутреннего экстремума]
Для того, чтобы точка $x_0$ была точкой экстремума функции
$f: U(x_0) \rightarrow \R$, необходимо выполнение одного из двух условий:
либо функция не дифференцируема в $x_0$, либо $f'(x_0) = 0$.
\end{statement}

\begin{statement}[Достаточные условия эксремума (в терминах первой производной)]
Кратко (не очень строго), если при переходе через точку производная меняет знак,
то экстремум есть, а если знак при этом не меняется, то экстремума нет.
\end{statement}

\begin{statement}[Достаточные условия экстремума (в терминах высших производных)]
Пусть функция $f: U(x_0) \rightarrow \R$ имеет в $x_0$ производные до порядка $n$
включительно ($n \ge 1$).

Если $f'(x_0) = f''(x_0) = \dots = f^{(n-1)} = 0$ и $f^{(n)} \ne 0$, то при
$n$ нечетном экстремума нет, а при четном экстремум есть, причем это строгий
локальный минимум, если $f^{(n)}(x_0) > 0$ и строгий локальный максимум,
если $f^{(n)}(x) < 0$.
\end{statement}

% Источники:
% 1. Зорич, страницы 208-278

\section
{
  Интегральное исчисление. Неопределенный интеграл. Определенный интеграл Римана.
  Формула Ньютона-Лейбница. Методы интегрирования.
}

\subsection{Неопределенный интеграл}
\begin{definition}
Неопределенный интеграл для функции $f(x)$ --- совокупность всех первообразных
данной функции.

Если функция $f(x)$ определена и непрерывна на $(a, b)$, а $F(x)$ --- первообразная
данной функции, т.е. $F'(x) = f(x), a < x < b$, тогда
$$
\int f(x) dx = F(x) + C, a < x < b.
$$
$C$ --- произвольная постоянная.
\end{definition}

\subsection{Определенный интеграл Римана}
\begin{definition}
Разбиением $P$ отрезка $[a, b], a < b$ называется конечная система точек
$x_0, x_1, \dots x_n$ этого отрезка, такая что $a = x_0 < x_1 < \dots < x_n = b$.

Отрезки $[x_{i-1}, x_i], (i=1, \dots, n)$ называются отрезками разбиения $P$.

Максимум $\lambda(P)$ из длин отрезков разбиения называется параметром разбиения $P$.
\end{definition}

\begin{definition}
Говорят, что имеется разбиение с отмеченными точками $(P, \xi)$ отрезка $[a, b]$,
если имеется разбиение $P$ отрезка $[a, b]$ и в каждом из отрезков $[x_{i-1}, x_i]$
разбиения $P$ выбрано по точке $\xi_i \in [x_{i-1}, x_i], (i=1, \dots, n)$.

Набор $(\xi_1, \dots, \xi_n)$ обозначается одной буквой $\xi$.
\end{definition}

\begin{definition}
Если функция определена на отрезке $[a, b]$, a $(P, \xi)$ --- разбиение с отмеченными
точками этого отрезка, то сумма
$$
\sigma(f, P, \xi) = \sum\limits_{i=1}^{n} f(\xi_i) \Delta x_i
$$
называется интегральной суммой функции $f$, соответствующей разбиению $(P, \xi)$
с отмеченными точками отрезка $[a, b]$, где $\Delta x_i = x_i - x_{i-1}$.
\end{definition}

\begin{definition}[Определенный интеграл Римана]
$$
\int\limits_{a}^{b} f(x) \, dx = \lim\limits_{\lambda(P) \rightarrow 0} \sum\limits_{i=0}^{n} f(\xi_i) \Delta x_i.
$$
\end{definition}

\subsection{Формула Ньютона-Лейбница}
\begin{theorem}
Если $f: [a, b] \rightarrow \R$ --- ограниченная функция, с конечным числом точек разрыва,
то $f \in \Ri [a, b]$ и
$$
\int\limits_a^b f(x)\,dx = F(b) - F(a),
$$
где $F: [a, b]$ --- любая из первообразных функции $f$ на $[a, b]$.
\end{theorem}

\subsection{Методы интегрирования}
\subsubsection{Интегрирование по частям}
$$
\int\limits_a^b u \cdot d v = (u\cdot v) \Bigl |_a^b - \int\limits_a^b v du
$$

Доказательство вытекает из формулы дифференцирования произведения.

\subsection{Замена переменной в интеграле}
\begin{theorem}
Если $\phi: [\alpha, \beta] \rightarrow [a, b]$ --- непрерывно дифференцируемое отображение
(т.е. имеет непрерывную производную) отрезка $\alpha < t < \beta$ в отрезок $a < x < b$ такое,
что $\varphi(\alpha) = a, \varphi(\beta) = b$, то при любой непрерывной на отрезке $[a, b]$ функции
$f(x)$ функция $f(\phi(t))\varphi'(t)$ --- непрерывна на отрезке и справедливо равенство:
$$
\int\limits_a^b f(x) \, dx = \int\limits_\alpha^\beta f(\varphi(t)) \varphi'(t)\, dt.
$$
\end{theorem}

% Источники:
% 1. https://ru.wikipedia.org/wiki/Неопределенный_интеграл
% 2. Зорич, страницы 385-427

\subsection{Функции многих переменных. Их предел и непрерывность.}

\begin{definition}
Условимся через $\R^m$ обозначать множество всех упорядоченных
наборов $(x^1, \dots, x^m)$ состоящих из $m$ действительных
чисел $x^i \in \R\,(i=1,\dots,m)$. Каждый набор обозначается буквой $x$
и называется точкой множества $R^m$.
\end{definition}

Расстояние между точками $x_1$ и $x_2$:
$$
d(x_1, x_2) = \sqrt{ \sum\limits_{i=1}^{m} (x_1^i - x_2^i)^2 }
$$

Функция $d: \R^m \times \R^m \rightarrow \R$ обладает свойствами:
\begin{enumerate}
\item $d(x_1, x_2) \ge 0$;
\item $d(x_1, x_2) = 0 \Rightarrow x_1 = x_2$;
\item $d(x_1, x_2) = d(x_2, x_1)$;
\item $d(x_1, x_3) \le d(x_1, x_2) + d(x_2, x_3).$
\end{enumerate}

\begin{definition}
d --- метрика или расстояние в $X$.
\end{definition}

\subsection{Предел}
\begin{definition}
Совокупность $\B$ множества $\B \subset X$ множества $X$ будем называть
базой в множестве $X$, если выполнены два условия:
\begin{enumerate}
\item $\forall B \in \B (B \ne \varnothing$)
\item $\forall B_1 \in \B, \forall B_2 \in \B \quad \exists B \in \B: B \subset B_1 \cup B_2$.
\end{enumerate}

Проще говоря, элементы совокупности $\B$ --- непустые множества и в пересечении любых
двух из них содержится элемент из той же совокупности.
\end{definition}

\begin{definition}
Точка $A \in \R^m$ называется пределом отображения $f: X \rightarrow \R^m$
по базе $\B$ в $X$, если для любой окрестности $V(A)$ этой точки, найдется элемент
$B \in \B$, образ которого $f(B)$ содержится в $V(A)$.
\end{definition}

$$
(\lim\limits_{\B} f(x) = A) \Leftrightarrow \forall V(A), \exists B \in \B: (f(B) \subset V(A).
$$

Или в более привычном виде:
$$
(\lim\limits_{\B} f(x) = A) \Leftrightarrow \forall \varepsilon, \exists B \in \B, \forall x \in B: d(f(x), A) < \varepsilon.
$$

\subsection{Непрерывность}
Пусть $E$ --- множество в пространстве $R^m$ и $f: E \rightarrow R^n$.

\begin{definition}
Функция $f: E \rightarrow R^n$ называется непрерывной в точке $a \in E$,
если для любой окрестности $V(f(a)) \in R^n$ найдется такая окрестность
$U(a) \in E$, что ее образ $f(U(a)) \subset V(f(a))$.
\end{definition}

Или тоже самое:
$$
\forall \varepsilon > 0, \exists \delta > 0, \forall x \in E: d(x, a) < \delta \Rightarrow d(f(x), f(a)) < \varepsilon.
$$

% Источники:
% Зорич, страницы 477-491

\chapter{Линейная алгебра и аналитическая геометрия}

\chapter{Дискретная математика}

\chapter{Теория вероятностей и математическая статистика}

\chapter{Теория случайных процессов и основы теории массового обслуживания}

\chapter{Методы оптимизации и основы теории управления}

\chapter{Метематическое моделирование}

\chapter{Механика}

\chapter{Численные методы}

\chapter{Программирование для ЭВМ}

\chapter{Программные и аппаратные средства информатики}

\chapter{Операционные системы и сети ЭВМ}

\chapter{Базы данных}

\end{document}

