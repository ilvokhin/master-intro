\documentclass[12pt]{report}

\usepackage{fullpage}
\usepackage{multicol, multirow}
\usepackage{tabularx}
\usepackage{standalone}
\usepackage{listings}
\usepackage{ulem}
\usepackage{amsmath}
\usepackage{amssymb}
\usepackage{pdfpages}
\usepackage[T1, T2A]{fontenc}
\usepackage[utf8]{inputenc}
\usepackage[russian]{babel}

\usepackage{indentfirst}
% setup  page fields
\usepackage{geometry}
\geometry{left=25mm}
\geometry{right=20mm}
\geometry{top=20mm}
\geometry{bottom=20mm}

\usepackage{titlesec}
\titleformat{\chapter}{\filcenter\bfseries}{\thechapter.}{1em}{}
\titleformat{\section}{\filcenter\bfseries}{\thesection}{1em}{}
\titleformat{\subsection}{\filcenter\bfseries}{\thesubsection}{1em}{}
\titleformat{\subsubsection}{\filcenter\bfseries}{\thesubsubsection}{1em}{}
\titleformat{\paragraph}{\filcenter\bfseries}{\paragraph}{1em}{}

\usepackage{tocloft}
% add dots for chapters
\renewcommand{\cftchapleader}{\cftdotfill{\cftdotsep}}
\renewcommand\cftchapfont{\mdseries}
\renewcommand\cftchappagefont{\mdseries}
% add dot after chapter number
\renewcommand{\cftchapaftersnum}{.}
% clickable toc
\usepackage{hyperref}

%
\usepackage{amsthm}
\theoremstyle{plain}
\newtheorem{theorem}{Теорема}[chapter]
\newtheorem{definition}[theorem]{Определение}

% Some usefull macro:
\newcommand{\R}{\mathbb R}

\begin{document}

\tableofcontents

\chapter{Математический анализ}

\section{Аксиоматика и общие свойства множества действительных чисел. Полнота множества действительных чисел.}
\begin{definition}
Множество $\R$ называется множеством действительных (вещественных) чисел,
а его элементы действительными (вещественными) числами, если выполняются следующие
условия (аксиомы действительных чисел):
\end{definition}

\subsection{Аксиомы сложения}
Определено отображение (операция сложения):
$$
  +: \R \times \R \rightarrow \R,
$$
которое каждой упорядоченной паре $(x, y)$ ставит в соответствие элемент
$x + y \in \R$, называемый суммой $x$ и $y$.
При этом выполнены следующие условия:

\begin{enumerate}

\item Существует нейтральный элемент $0$ (в случае сложения называется нулем) такой,
что для любого $x\in \R$:
$$
x + 0 = 0 + x = x.
$$

\item Для любого элемента $x \in \R$ имеется элемент $-x \in \R$,
называемый противоположным элементом к $x$, такой, что:
$$
x + (-x) = (-x) + x = 0.
$$

\item Операция $+$ --- ассоциативна, т.е. для любых $x, y, z \in \R$ выполнено:
$$
(x + y) + z = x + (y + z).
$$

\item Операция $+$ --- коммутативна, т.е. для любых $x, y \in \R$ выполнено:
$$
x + y = y + x.
$$

\end{enumerate}

Если на каком-то множестве $G$ выполнены аксиомы 1-3, говорят, что $G$ --- группа.
Если операцию называют сложением, то группа называется аддитивной.
Если известно, что операция коммутативна (т.е. выполнена аксиома 4), говорят, что
группа коммутативна или абелева.

\subsection{Аксиомы умножения}
Определено отображение (операция умножения):
$$
\cdot: \R \times \R \rightarrow \R,
$$
составляющая каждой упорядоченной паре $(x, y) x, y \in \R$
некоторый элемент $x \cdot y \in \R$, называемый произведением $x$ и $y$,
причем так, что выполнены следующие условия:

\begin{enumerate}
\item Существует нейтральный элемент $1 \in R \setminus 0$,
в случае умножения называемый единицей, так что
$$
x \cdot 1 = 1 \cdot x = x.
$$

\item Для $\forall x \in \R \setminus 0$ имеется элемент $x^{-1}$, называемый
обратным, такой, что:
$$
x \cdot x^{-1} = x^{-1} \cdot x = 1.
$$

\item Операция $\cdot$ --- ассоциативна, т.е. для любых $x, y, z \in \R$:
$$
x\cdot (y \cdot z) = (x \cdot y) \cdot z.
$$

\item Операция $\cdot$ --- коммутативна, т.е. для $x, y \in \R$:
$$
x \cdot y = y \cdot x.
$$

\end{enumerate}

Связь сложения и умножения: умножение дистрибутивно по отношению к сложению, т.е.
$\forall x, y, z \in \R$:
$$
(x + y) \cdot z = xz + xy.
$$

Если на множестве $G$ действуют две операции, удовлетворяющие всем перечисленным выше аксиомам,
то $G$ называется алгебраическим полем или просто полем.

\subsection{Аксиомы порядка}
Между элементами $\R$ имеется отношение $\le$, т.е. для элементов $x, y \in \R$ установлено,
выполняется ли $x \le y$ или нет. При этом должны удовлетворятся следующие условия:

\begin{enumerate}
\item $\forall x \in \R: x \le x$.
\item $(x \le y) \land (y \le x) \implies (x = y)$.
\item $(x \le y) \land (y \le z) \implies (x \le z)$.
\item $\forall x \in \R, \forall y \in \R: (x \le y) \lor (y \le x)$.
\end{enumerate}

Отношение $\le$ называется отношением неравенства в $\R$.

Множество, в котором выполняются аксиомы 1-3 называют частично упорядоченным.
Если сверх того выполнена аксиома 4, т.е. любые два элемента множества сравнимы, то
множество называется линейно упорядоченным.

Связь сложения и порядка в $\R$: $\forall x, y, z \in \R: (x \le y) \implies (x + z \le y + z)$.
Связь умножения и порядка в $\R$: $(0 \le x) \land (0 \le y) \implies (0 \le x \cdot y)$.

\subsection{Аксиома (полноты) непрерывности}
Если $X$ и $Y$ --- непустые подмножества $\R$, обладающие тем свойством,
что для любых элементов $x \in X$ и $y \in Y$ выполнено $x \le y$,
то существует такое $c: x \le c \le y$ для любых элементов $x \in X$ и $y \in Y$.

\subsection{Общие свойства множества действительных чисел}

\subsubsection{Следствия из аксиом сложения}
\begin{enumerate}
\item В множестве действительных чисел имеется только один нуль.
\item В множестве действительных чисел у каждого элемента имеется
  только один противоположный элемент.
\item Уравнение $a + x = b$ в $\R$ имеет и притом единственное решение:
  $x = b + (-a)$.
\end{enumerate}

\subsubsection{Следствия из аксиом умножения}
\begin{enumerate}
\item В множестве действительных чисел имеется только одна единица.
\item Для каждого числа $x \ne 0$ имеется только один обратный элемент $x^{-1}$.
\item Уравнение $a \cdot x = b$ при $a \in R \setminus 0$ имеет и притом единственное
  решение $x = b \cdot a^{-1}$.
\end{enumerate}

\subsubsection{Следствия из аксиом связи сложения и умножения}
\begin{enumerate}
\item $\forall x \in \R: x \cdot 0 = 0 \cdot x = 0$.
\item $(x \cdot y = 0) \implies (x = 0) \lor (y = 0)$.
\item $\forall x \in \R: -x = (-1) \cdot x$.
\item $\forall x \in \R: (-1) \cdot (-x) = x$.
\item $\forall x \in \R: (-x) \cdot (-x) = x \cdot x$.
\end{enumerate}

\subsubsection{Следствия аксиом порядка}
\begin{enumerate}
\item $\forall x, y \in \R$ всегда имеет место в точности одно из отношений:
$$
x < y, \quad x = y, \quad x > y.
$$
\item $\forall x, y, z \in \R$:
  \begin{gather*}
    (x < y) \land (y \le z) \implies (x < z), \\
    (x \le y) \land (y < z) \implies (x < z).
  \end{gather*}
\end{enumerate}

\subsubsection{Следствия аксиом связи порядка со сложением и умножением}
\begin{enumerate}
\item $\forall x, y, z, w \in \R$:
  \begin{gather*}
    (x < y) \implies (x + z) < (y + z), \\
    (0 < x) \implies (-x < 0), \\
    (x \le y) \land (z \le w) \implies (x + z \le y + w), \\
    (x \le y) \land  (z < w) \implies (x + z < y + w).
  \end{gather*}
\item $\forall x, y, z \in \R$:
  \begin{gather*}
    (0 < x) \land (0 < y) \implies (0 < xy), \\
    (x < 0) \land (y < 0) \implies (0 < xy), \\
    (x < 0) \land (0 < y) \implies (xy < 0), \\
    (x < y) \land (0 < z) \implies (xz < yz), \\
    (x < y) \land (z < 0) \implies (yz < xz).
  \end{gather*}
\item $0 < 1$.
\item $(0 < x) \implies (0 < x^{-1})$ и $(0 < x) \land (x < y) \implies (0 < y^{-1}) \land (y^{-1} < x^{-1}).$
\end{enumerate}

% Источник: Зорич, страницы 41 - 50.
% Там можно найти доказательства некоторых следствий.

\chapter{Линейная алгебра и аналитическая геометрия}

\chapter{Дискретная математика}

\chapter{Теория вероятностей и математическая статистика}

\chapter{Теория случайных процессов и основы теории массового обслуживания}

\chapter{Методы оптимизации и основы теории управления}

\chapter{Метематическое моделирование}

\chapter{Механика}

\chapter{Численные методы}

\chapter{Программирование для ЭВМ}

\chapter{Программные и аппаратные средства информатики}

\chapter{Операционные системы и сети ЭВМ}

\chapter{Базы данных}

\end{document}

